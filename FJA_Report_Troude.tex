\documentclass{report}
% Change "article" to "report" to get rid of page number on title page
\usepackage{amsmath,amsfonts,amsthm,amssymb}
\usepackage{subcaption}
\usepackage{caption}
\usepackage{setspace}
\usepackage{fancyhdr}
\usepackage{lastpage}
\usepackage{extramarks}
\usepackage{chngpage}
\usepackage{soul}
\usepackage[usenames,dvipsnames]{color}
\usepackage{graphicx,float,wrapfig}
\usepackage{ifthen}
\usepackage{listings}
\usepackage{courier}


%   !!!!!!!!!!!!!!!!!!!!!!!!!!!!!!!!!!!!!!!!!!!!!!!!!!!!!
%
%   Here put your info (name, due date, title etc).
%   the rest should be left unchanged.
%
%
%
%   !!!!!!!!!!!!!!!!!!!!!!!!!!!!!!!!!!!!!!!!!!!!!!!!!!!!!


% Homework Specific Information
\newcommand{\hmwkTitle}{FastJet algorithm}
\newcommand{\hmwkSubTitle}{}
\newcommand{\hmwkDueDate}{October, 2019}
\newcommand{\hmwkClass}{Proseminar}
\newcommand{\hmwkClassTime}{}
\newcommand{\hmwkClassInstructor}{}


\newcommand{\hmwkAuthorName}{Virgile TROUDE}
%
%

% In case you need to adjust margins:
\topmargin=-0.45in      %
\evensidemargin=0in     %
\oddsidemargin=0in      %
\textwidth=6.5in        %
\textheight=9.5in       %
\headsep=0.25in         %

% This is the color used for  comments below
\definecolor{MyDarkGreen}{rgb}{0.0,0.4,0.0}

% For faster processing, load Matlab syntax for listings
\lstloadlanguages{Matlab}%
\lstset{language=Matlab,                        % Use MATLAB
        frame=single,                           % Single frame around code
        basicstyle=\small\ttfamily,             % Use small true type font
        keywordstyle=[1]\color{Blue}\bf,        % MATLAB functions bold and blue
        keywordstyle=[2]\color{Purple},         % MATLAB function arguments purple
        keywordstyle=[3]\color{Blue}\underbar,  % User functions underlined and blue
        identifierstyle=,                       % Nothing special about identifiers
                                                % Comments small dark green courier
        commentstyle=\usefont{T1}{pcr}{m}{sl}\color{MyDarkGreen}\small,
        stringstyle=\color{Purple},             % Strings are purple
        showstringspaces=false,                 % Don't put marks in string spaces
        tabsize=3,                              % 5 spaces per tab
        %
        %%% Put standard MATLAB functions not included in the default
        %%% language here
        morekeywords={xlim,ylim,var,alpha,factorial,poissrnd,normpdf,normcdf},
        %
        %%% Put MATLAB function parameters here
        morekeywords=[2]{on, off, interp},
        %
        %%% Put user defined functions here
        morekeywords=[3]{FindESS, homework_example},
        %
        morecomment=[l][\color{Blue}]{...},     % Line continuation (...) like blue comment
        numbers=left,                           % Line numbers on left
        firstnumber=1,                          % Line numbers start with line 1
        numberstyle=\tiny\color{Blue},          % Line numbers are blue
        stepnumber=1                        % Line numbers go in steps of 5
        }

% Setup the header and footer
\pagestyle{fancy}                                                       %
\lhead{\hmwkAuthorName}                                                 %
%\chead{\hmwkClass\ (\hmwkClassInstructor\ \hmwkClassTime): \hmwkTitle}  %
\rhead{\hmwkClass\ : \hmwkTitle}  %
%\rhead{\firstxmark}                                                     %
\lfoot{\lastxmark}                                                      %
\cfoot{}                                                                %
\rfoot{Page\ \thepage\ of\ \protect\pageref{LastPage}}                  %
\renewcommand\headrulewidth{0.4pt}                                      %
\renewcommand\footrulewidth{0.4pt}                                      %

% This is used to trace down (pin point) problems
% in latexing a document:
%\tracingall

%%%%%%%%%%%%%%%%%%%%%%%%%%%%%%%%%%%%%%%%%%%%%%%%%%%%%%%%%%%%%
% Some tools
\newcommand{\enterProblemHeader}[1]{\nobreak\extramarks{#1}{#1 continued on next page\ldots}\nobreak%
                                    \nobreak\extramarks{#1 (continued)}{#1 continued on next page\ldots}\nobreak}%
\newcommand{\exitProblemHeader}[1]{\nobreak\extramarks{#1 (continued)}{#1 continued on next page\ldots}\nobreak%
                                   \nobreak\extramarks{#1}{}\nobreak}%

\newlength{\labelLength}
\newcommand{\labelAnswer}[2]
  {\settowidth{\labelLength}{#1}%
   \addtolength{\labelLength}{0.25in}%
   \changetext{}{-\labelLength}{}{}{}%
   \noindent\fbox{\begin{minipage}[c]{\columnwidth}#2\end{minipage}}%
   \marginpar{\fbox{#1}}%

   % We put the blank space above in order to make sure this
   % \marginpar gets correctly placed.
   \changetext{}{+\labelLength}{}{}{}}%

\setcounter{secnumdepth}{0}
\newcommand{\homeworkProblemName}{}%
\newcounter{homeworkProblemCounter}%
\newenvironment{homeworkProblem}[1][Problem \arabic{homeworkProblemCounter}]%
  {\stepcounter{homeworkProblemCounter}%
   \renewcommand{\homeworkProblemName}{#1}%
   \section{\homeworkProblemName}%
   \enterProblemHeader{\homeworkProblemName}}%
  {\exitProblemHeader{\homeworkProblemName}}%

\newcommand{\problemAnswer}[1]
  {\noindent\fbox{\begin{minipage}[c]{\columnwidth}#1\end{minipage}}}%

\newcommand{\problemLAnswer}[1]
  {\labelAnswer{\homeworkProblemName}{#1}}

\newcommand{\homeworkSectionName}{}%
\newlength{\homeworkSectionLabelLength}{}%
\newenvironment{homeworkSection}[1]%
  {% We put this space here to make sure we're not connected to the above.
   % Otherwise the changetext can do funny things to the other margin

   \renewcommand{\homeworkSectionName}{#1}%
   \settowidth{\homeworkSectionLabelLength}{\textwidth}%
   \addtolength{\homeworkSectionLabelLength}{0.25in}%
   \changetext{}{-\homeworkSectionLabelLength}{}{}{}%
   \subsection{\homeworkSectionName}%
   \enterProblemHeader{\homeworkProblemName\ [\homeworkSectionName]}}%
  {\enterProblemHeader{\homeworkProblemName}%

   % We put the blank space above in order to make sure this margin
   % change doesn't happen too soon (otherwise \sectionAnswer's can
   % get ugly about their \marginpar placement.
   \changetext{}{+\homeworkSectionLabelLength}{}{}{}}%

\newcommand{\sectionAnswer}[1]
  {% We put this space here to make sure we're disconnected from the previous
   % passage

   \noindent\fbox{\begin{minipage}[c]{\columnwidth}#1\end{minipage}}%
   \enterProblemHeader{\homeworkProblemName}\exitProblemHeader{\homeworkProblemName}%
   \marginpar{\fbox{\homeworkSectionName}}%

   % We put the blank space above in order to make sure this
   % \marginpar gets correctly placed.
   }%

%%% I think \captionwidth (commented out below) can go away
%%%
%% Edits the caption width
%\newcommand{\captionwidth}[1]{%
%  \dimen0=\columnwidth   \advance\dimen0 by-#1\relax
%  \divide\dimen0 by2
%  \advance\leftskip by\dimen0
%  \advance\rightskip by\dimen0
%}

% Includes a figure
% The first parameter is the label, which is also the name of the figure
%   with or without the extension (e.g., .eps, .fig, .png, .gif, etc.)
%   IF NO EXTENSION IS GIVEN, LaTeX will look for the most appropriate one.
%   This means that if a DVI (or PS) is being produced, it will look for
%   an eps. If a PDF is being produced, it will look for nearly anything
%   else (gif, jpg, png, et cetera). Because of this, when I generate figures
%   I typically generate an eps and a png to allow me the most flexibility
%   when rendering my document.
% The second parameter is the width of the figure normalized to column width
%   (e.g. 0.5 for half a column, 0.75 for 75% of the column)
% The third parameter is the caption.
\newcommand{\scalefig}[3]{
  \begin{figure}[ht!]
    % Requires \usepackage{graphicx}
    \centering
    \includegraphics[width=#2\columnwidth]{#1}
    %%% I think \captionwidth (see above) can go away as long as
    %%% \centering is above
    %\captionwidth{#2\columnwidth}%
    \caption{#3}
    \label{#1}
  \end{figure}}

% Includes a MATLAB script.
% The first parameter is the label, which also is the name of the script
%   without the .m.
% The second parameter is the optional caption.
\newcommand{\matlabscript}[2]
  {\begin{itemize}\item[]\lstinputlisting[caption=#2,label=#1]{#1.m}\end{itemize}}

%%%%%%%%%%%%%%%%%%%%%%%%%%%%%%%%%%%%%%%%%%%%%%%%%%%%%%%%%%%%%


%%%%%%%%%%%%%%%%%%%%%%%%%%%%%%%%%%%%%%%%%%%%%%%%%%%%%%%%%%%%%
% Make title
%\title{\vspace{2in}\textmd{\textbf{\hmwkClass:\ \hmwkTitle\ifthenelse{\equal{\hmwkSubTitle}{}}{}{\\\hmwkSubTitle}}}\\\normalsize\vspace{0.1in}\small{Due\ on\ \hmwkDueDate}\\\vspace{0.1in}\large{\textit{\hmwkClassInstructor\ \hmwkClassTime}}\vspace{3in}}
\title{\vspace{2in}\textmd{\textbf{\hmwkClass:\ \hmwkTitle\ifthenelse{\equal{\hmwkSubTitle}{}}{}{\\\hmwkSubTitle}}}\\\normalsize\vspace{0.1in}\small{Due\ on\ \hmwkDueDate}\\\vspace{0.1in}\large{\textit{ \hmwkClassTime}}\vspace{3in}}
\date{}
\author{\textbf{\hmwkAuthorName}}
%%%%%%%%%%%%%%%%%%%%%%%%%%%%%%%%%%%%%%%%%%%%%%%%%%%%%%%%%%%%%

\begin{document}
\begin{spacing}{1.1}
\maketitle
% Uncomment the \tableofcontents and \newpage lines to get a Contents page
% Uncomment the \setcounter line as well if you do NOT want subsections
%       listed in Contents
%\setcounter{tocdepth}{1}
\newpage
\tableofcontents
\newpage

% When problems are long, it may be desirable to put a \newpage or a
% \clearpage before each homeworkProblem environment

\newpage

\begin{homeworkProblem}[Test algorithm]

  \begin{homeworkSection}{k$_t$-algorithm}
	
	\begin{figure}[H]
    	\centering
	\begin{subfigure}{0.45\textwidth}
		\centering
    		\includegraphics[width=\textwidth]{test_kt_p}
		\caption{Initial configuration}
		\label{test_kt_p}
   	\end{subfigure}
	\begin{subfigure}{0.45\textwidth}
		\centering
    		\includegraphics[width=\textwidth]{test_kt_j}
		\caption{Jet configuration}
		\label{test_kt_j}
   	\end{subfigure}
    	\caption{Test of the k$_t$-algorithm, where the initial configuration should give a final configuration with three jets for R=1}
    	\label{test_kt}
  	\end{figure}

	\begin{figure}[H]
    	\centering
	\begin{subfigure}{0.45\textwidth}
		\centering
    		\includegraphics[width=\textwidth]{kt_t0}
		\caption{}
		\label{kt_t0}
   	\end{subfigure}
	\begin{subfigure}{0.45\textwidth}
		\centering
    		\includegraphics[width=\textwidth]{kt_t1}
		\caption{}
		\label{kt_t1}
   	\end{subfigure}
	\begin{subfigure}{0.45\textwidth}
		\centering
    		\includegraphics[width=\textwidth]{kt_t2}
		\caption{}
		\label{kt_t2}
   	\end{subfigure}
	\begin{subfigure}{0.45\textwidth}
		\centering
    		\includegraphics[width=\textwidth]{kt_t3}
		\caption{}
		\label{kt_t3}
   	\end{subfigure}
	\begin{subfigure}{0.45\textwidth}
		\centering
    		\includegraphics[width=\textwidth]{kt_t4}
		\caption{}
		\label{kt_t4}
   	\end{subfigure}
	\begin{subfigure}{0.45\textwidth}
		\centering
    		\includegraphics[width=\textwidth]{kt_t5}
		\caption{}
		\label{kt_t5}
   	\end{subfigure}
    	\caption{Evolution of the momentum configuration during the k$_t$-algorithm with R=1.5 and N=50 momentum, from (a) to (f)}
    	\label{fja_t}
  	\end{figure}

  \end{homeworkSection}

  \begin{homeworkSection}{FastJet algorithm}
	
	\begin{figure}[H]
    	\centering
	\begin{subfigure}{0.45\textwidth}
		\centering
    		\includegraphics[width=\textwidth]{test_fja_p}
		\caption{Initial configuration}
		\label{test_fja_p}
   	\end{subfigure}
	\begin{subfigure}{0.45\textwidth}
		\centering
    		\includegraphics[width=\textwidth]{test_fja_j}
		\caption{Jet configuration}
		\label{test_fja_j}
   	\end{subfigure}
    	\caption{Test of the FastJet algorithm, where the initial configuration should give a final configuration with three jets for R=1}
    	\label{test_fja}
  	\end{figure}

	\begin{figure}[H]
    	\centering
	\begin{subfigure}{0.45\textwidth}
		\centering
    		\includegraphics[width=\textwidth]{fja_t0}
		\caption{}
		\label{fja_t0}
   	\end{subfigure}
	\begin{subfigure}{0.45\textwidth}
		\centering
    		\includegraphics[width=\textwidth]{fja_t1}
		\caption{}
		\label{fja_t1}
   	\end{subfigure}
	\begin{subfigure}{0.45\textwidth}
		\centering
    		\includegraphics[width=\textwidth]{fja_t2}
		\caption{}
		\label{fja_t2}
   	\end{subfigure}
	\begin{subfigure}{0.45\textwidth}
		\centering
    		\includegraphics[width=\textwidth]{fja_t3}
		\caption{}
		\label{fja_t3}
   	\end{subfigure}
	\begin{subfigure}{0.45\textwidth}
		\centering
    		\includegraphics[width=\textwidth]{fja_t4}
		\caption{}
		\label{fja_t4}
   	\end{subfigure}
	\begin{subfigure}{0.45\textwidth}
		\centering
    		\includegraphics[width=\textwidth]{fja_t5}
		\caption{}
		\label{fja_t5}
   	\end{subfigure}
    	\caption{Evolution of the momentum configuration during the FastJet algorithm with R=1.5 and N=50 momentum, from (a) to (f)}
    	\label{fja_t}
  	\end{figure}

  \end{homeworkSection}

\end{homeworkProblem}

\newpage

\begin{homeworkProblem}[IRC-safety]

  \begin{homeworkSection}{k$_t$-algorithm}
	
	\begin{figure}[H]
    	\centering
	\begin{subfigure}{\textwidth}
		\centering
    		\includegraphics[width=\textwidth]{kt_ir_n}
		\caption{Observable: Number of jet (N)}
		\label{kt_ir_n}
   	\end{subfigure}
	\begin{subfigure}{\textwidth}
		\centering
    		\includegraphics[width=\textwidth]{kt_ir_e}
		\caption{Observable: Energy of the most energetic jet (E)}
		\label{kt_ir_e}
   	\end{subfigure}
    	\caption{IR-safety of k$_t$-algorithm for N=100 momentum and R=1}
    	\label{kt_ir}
  	\end{figure}

	\begin{figure}[H]
    	\centering
	\begin{subfigure}{\textwidth}
		\centering
    		\includegraphics[width=\textwidth]{kt_c_n}
		\caption{Observable: Number of jet (N)}
		\label{kt_c_n}
   	\end{subfigure}
	\begin{subfigure}{\textwidth}
		\centering
    		\includegraphics[width=\textwidth]{kt_c_e}
		\caption{Observable: Energy of the most energetic jet (E)}
		\label{kt_c_e}
   	\end{subfigure}
    	\caption{C-safety of k$_t$-algorithm for N=100 momentum and R=1}
    	\label{kt_c}
  	\end{figure}

	\begin{figure}[H]
    	\centering
	\begin{subfigure}{0.45\textwidth}
		\centering
    		\includegraphics[width=\textwidth]{kt_p0}
		\caption{Initial momentum configuration}
		\label{kt_p0}
   	\end{subfigure}
	\begin{subfigure}{0.45\textwidth}
		\centering
    		\includegraphics[width=\textwidth]{kt_j0}
		\caption{Initial jet configuration}
		\label{kt_j0}
   	\end{subfigure}
	\begin{subfigure}{0.45\textwidth}
		\centering
    		\includegraphics[width=\textwidth]{kt_p_k=3_16e_01}
		\caption{Momentum after splitting and emission with k=3.16$\times$10$^{-1}$}
		\label{kt_p_k1}
   	\end{subfigure}
	\begin{subfigure}{0.45\textwidth}
		\centering
    		\includegraphics[width=\textwidth]{kt_j_k=3_16e_01}
		\caption{Jet after splitting and emission with k=3.16$\times$10$^{-1}$}
		\label{kt_j_k1}
   	\end{subfigure}
	\begin{subfigure}{0.45\textwidth}
		\centering
    		\includegraphics[width=\textwidth]{kt_p_k=5_81e_03}
		\caption{Momentum after splitting and emission with k=5.81$\times$10$^{-3}$}
		\label{kt_p_k2}
   	\end{subfigure}
	\begin{subfigure}{0.45\textwidth}
		\centering
    		\includegraphics[width=\textwidth]{kt_j_k=5_81e_03}
		\caption{Jet after splitting and emission with k=5.81$\times$10$^{-3}$}
		\label{kt_j_k2}
   	\end{subfigure}
    	\caption{Visual representation of the IRC-safety for N=5 initial momentum when 10 splitting followed by 10 emissions happens with a coefficent k, for the k$_t$-algorithm with R=1}
    	\label{kt_irc}
  	\end{figure}
  
  \end{homeworkSection}

  \begin{homeworkSection}{FastJet algorithm}

	\begin{figure}[H]
    	\centering
	\begin{subfigure}{\textwidth}
		\centering
    		\includegraphics[width=\textwidth]{fja_ir_n}
		\caption{Observable: Number of jet (N)}
		\label{fja_ir_n}
   	\end{subfigure}
	\begin{subfigure}{\textwidth}
		\centering
    		\includegraphics[width=\textwidth]{fja_ir_e}
		\caption{Observable: Energy of the most energetic jet (E)}
		\label{fja_ir_e}
   	\end{subfigure}
    	\caption{IR-safety of FastJet algorithm for N=100 momentum and R=1}
    	\label{fja_ir}
  	\end{figure}

	\begin{figure}[H]
    	\centering
	\begin{subfigure}{\textwidth}
		\centering
    		\includegraphics[width=\textwidth]{fja_c_n}
		\caption{Observable: Number of jet (N)}
		\label{fja_c_n}
   	\end{subfigure}
	\begin{subfigure}{\textwidth}
		\centering
    		\includegraphics[width=\textwidth]{fja_c_e}
		\caption{Observable: Energy of the most energetic jet (E)}
		\label{fja_c_e}
   	\end{subfigure}
    	\caption{C-safety of FastJet algorithm for N=100 momentum and R=1}
    	\label{fja_c}
  	\end{figure}

	\begin{figure}[H]
    	\centering
	\begin{subfigure}{0.45\textwidth}
		\centering
    		\includegraphics[width=\textwidth]{fja_p0}
		\caption{Initial momentum configuration}
		\label{fja_p0}
   	\end{subfigure}
	\begin{subfigure}{0.45\textwidth}
		\centering
    		\includegraphics[width=\textwidth]{fja_j0}
		\caption{Initial jet configuration}
		\label{fja_j0}
   	\end{subfigure}
	\begin{subfigure}{0.45\textwidth}
		\centering
    		\includegraphics[width=\textwidth]{fja_p_k=3_16e_01}
		\caption{Momentum after splitting and emission with k=3.16$\times$10$^{-1}$}
		\label{fja_p_k1}
   	\end{subfigure}
	\begin{subfigure}{0.45\textwidth}
		\centering
    		\includegraphics[width=\textwidth]{fja_j_k=3_16e_01}
		\caption{Jet after splitting and emission with k=3.16$\times$10$^{-1}$}
		\label{fja_j_k1}
   	\end{subfigure}
	\begin{subfigure}{0.45\textwidth}
		\centering
    		\includegraphics[width=\textwidth]{fja_p_k=5_81e_03}
		\caption{Momentum after splitting and emission with k=5.81$\times$10$^{-3}$}
		\label{fja_p_k2}
   	\end{subfigure}
	\begin{subfigure}{0.45\textwidth}
		\centering
    		\includegraphics[width=\textwidth]{fja_j_k=5_81e_03}
		\caption{Jet after splitting and emission with k=5.81$\times$10$^{-3}$}
		\label{fja_j_k2}
   	\end{subfigure}
    	\caption{Visual representation of the IRC-safety for N=5 initial momentum when 10 splitting followed by 10 emissions happens with a coefficent k, for the FastJet algorithm with R=1}
    	\label{fja_irc}
  	\end{figure}

  \end{homeworkSection}

\end{homeworkProblem}

\begin{homeworkProblem}[Time Complexity]

  \begin{homeworkSection}{k$_t$-algorithm}

	\begin{figure}[H]
	\centering
	\includegraphics[width=\textwidth]{kt_complexity}
    	\caption{Time complexity of the k$_t$-algorithm with N=100 momentum and R=1}
    	\label{kt_complexity}
  	\end{figure}

  \end{homeworkSection}

  \begin{homeworkSection}{FastJet algorithm}

	\begin{figure}[H]
	\centering
	\includegraphics[width=\textwidth]{fja_complexity}
    	\caption{Time complexity of the FastJet algorithm with N=100 momentum and R=1}
    	\label{fja_complexity}
  	\end{figure}

  \end{homeworkSection}

\end{homeworkProblem}

\newpage

\begin{thebibliography}{99}
\bibitem{}
\end{thebibliography}

\end{spacing}
\end{document}

%%%%%%%%%%%%%%%%%%%%%%%%%%%%%%%%%%%%%%%%%%%%%%%%%%%%%%%%%%%%%
